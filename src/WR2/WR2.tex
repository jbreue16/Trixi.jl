\documentclass[a4paper,DIV8,10pt]{scrartcl}

\usepackage[T1]{fontenc}
\usepackage{geometry}
\geometry{left=3cm,right=3cm,top=3cm,bottom=3cm,bindingoffset=5mm}

  \usepackage{german}    % fuer die deutschen Trennmuster
  \usepackage{ngerman} % entsprechend fuer die neue Rechtschreibung
  \usepackage[latin1]{inputenc} % falls Sie Umlaute in den Quellen verwenden wollen
  \usepackage{amsmath}   % enthaelt nuetzliche Makros fuer Mathematik
  \usepackage{amsthm}    % fuer Saetze, Definitionen, Beweise, etc.
  \usepackage{amsfonts}  % spezielle AMS-Mathematik-Fonts
  \usepackage{relsize}   % fuer \smaller 
  \usepackage{tikz}      % fuer Graphiken
  \usepackage[]{ragged2e}
  \usepackage{hyperref}
  \usepackage{url}
  \usepackage{amsmath}
\usepackage{amssymb}
\usepackage{mathtools}
\usepackage{graphicx}
\usepackage{float}
\usepackage{stmaryrd}

 \def\Gleich#1{\stackrel{\text{\makebox[0pt]{#1}}}{=}} 
  
  \newcommand{\N}{\ensuremath{\mathbb{N}}}   % natuerliche Zahlen
  \newcommand{\Z}{\ensuremath{\mathbb{Z}}}   % ganze Zahlen
  \newcommand{\Q}{\ensuremath{\mathbb{Q}}}   % rationale Zahlen
  \newcommand{\R}{\ensuremath{\mathbb{R}}}   % reelle Zahlen
    \newcommand\scalemath[2]{\scalebox{#1}{\mbox{\ensuremath{\displaystyle #2}}}}
\definecolor{darkcerulean}{rgb}{0.03, 0.27, 0.49}

  \theoremstyle{definition}
  \newtheorem{satz}{Satz}[section]
  \newtheorem{lemma}[satz]{Lemma}
  \newtheorem{korollar}[satz]{Korollar}
  \newtheorem{definition}[satz]{Definition}
  \newtheorem{bemerkung}[satz]{Bemerkung}
  \newtheorem{aufgabe}[satz]{Aufgabe}
  \newenvironment{beweis}%
    {\begin{proof}[Beweis]}
    {\end{proof}}
  \newtheorem{beispiel}[satz]{Beispiel}

  \renewcommand{\labelitemi}{--}             % aendert die Symbole bei unnumerierten Aufzaehlungen
  \makeatletter                              % Fussnote ohne Symbol
    \def\blfootnote{\xdef\@thefnmark{}\@footnotetext}
  % Titel des Handouts
  %   #1 Name des Vortragenden
  %   #2 email-Adresse 
  %   #3 Datum des Vortrags
  %   #4 Titel des Vortrags
  \newcommand{\handouttitle}[4]
   {\begin{center}
      \large #4
    \end{center}

    \bigskip

    \noindent
    #1 (\textsf{#2})
    \hfill
    #3

  
    \noindent
    \rule{\linewidth}{.5pt}

    \bigskip

    \@afterindentfalse\@afterheading
   }
  \makeatother
  \renewcommand{\sectfont}{\normalfont}       % aendert den Font fuer Ueberschriften



%----------------------------------------------------------------------------------------------------------------------------------------------
\begin{document}

\handouttitle
{Wissenschaftliches Rechnen II}
{SS21}
{\today}

%-------------------------------------------------------------------------------------------
% Inhaltsverzeichnis
%-------------------------------------------------------------------------------------------

% Erstellung des Inhaltsverzeichnisses auf einer neuen Seite
\tableofcontents

% Umstellung der Seitennummerierung auf kleine r�mische Ziffern
\pagenumbering{roman}


%% F�r das Einsetzen einer Leerseite soll dieser Code auskommentiert werden
%\newpage
%\thispagestyle{empty}
%\hspace{1cm}

\newpage

% Umstellung der Seitennummerierung auf arabische Ziffern
\pagenumbering{arabic}

\section{Meilenstein 1}

\textbf{Aufgabe 1}
\\
Formulate the compressible Euler equations in divergence form 
\begin{align*}
	u_t+f(u)_x + g(u)_y =0 
\end{align*}
and explicitly state the fluxes f, g.
\\
\\
\textbf{L�sung:}

Die kompressiblen Euler Gleichungen sind gegeben durch:
\begin{align*}
	\varrho_t+ \nabla \  \circ \ (\varrho v)&=0\\
	(\varrho v)_t+ \nabla \  \circ \ (\varrho vv^T + pI)&=0\\
	E_t + \nabla \ \circ \ [(E+p)v]&=0
\end{align*}
Da $v=(v_1,v_2)^T$ gilt, sind die Gleichungen �quivalent zu
\begin{align*}
	\varrho_t+ (\varrho v_1)_x + (\varrho v_1)_y  &=0\\
	(\varrho v_1)_t+ (\varrho v_1^2 + p)_x + (\varrho v_1 v_2)_y &=0\\
	(\varrho v_2)_t+ (\varrho v_1 v_2 )_x + (\varrho v_2^2 + p)_y &=0\\
	E_t + [(E+p)v_1]_x + [(E+p)v_2]_y&=0
\end{align*}
Damit l�sst sich die Divergenzform aufstellen:
\begin{align*}
	\left(\begin{array}{c} \varrho \\ \varrho v_1 \\ \varrho v_2 \\ E \end{array}\right)_t	+  \left(\begin{array}{c} \varrho v_1 \\ \varrho v_1^2 + p \\ \varrho v_1 v_2 \\ (E+p)v_1 \end{array}\right)_x +  \left(\begin{array}{c} \varrho v_2 \\ \varrho v_1 v_2 \\ \varrho v_2^2+p \\ (E+p)v_2 \end{array}\right)_y = 0
\end{align*}

\qed


\textbf{Aufgabe 2}
\\
Write the equations in their quasi-linear form
\begin{align*}
	u_t+Au_x + Bu_y =0 
\end{align*}
and find the eigenvalues of flux Jacobian matrices A, B. Explain why
the maximum eigenvalue is important in the explicit time integration.
\\
\\
\textbf{L�sung:}
Sei $u=(u_1,u_2,u_3,u_4)^T=(\varrho , \varrho v_1, \varrho v_2, E)^T$. \\
Dann gilt 
\begin{align*}
	p&=(\gamma -1)(E+\frac{\varrho}{2} ||v||^2)
	=(\gamma -1)(E+\frac{\varrho}{2} (v_1^2+v_2^2))
	=(\gamma -1)(u_4+ \frac{u_2^2}{2 u_1}+\frac{u_3^2}{2 u_1} )
\end{align*}
und 
\begin{align*}
	f=\left(\begin{array}{c} \varrho v_1 \\ \varrho v_1^2 + p \\ \varrho v_1 v_2 \\ (E+p)v_1 \end{array}\right)= \left(\begin{array}{c} u_2 \\ \frac{u_2^2}{u_1} + p \\ \frac{u_2 u_3}{u_1} \\ \frac{u_2}{u_1} (u_4+p) \end{array}\right)\\
	g=\left(\begin{array}{c} \varrho v_2 \\ \varrho v_1 v_2 \\ \varrho v_2^2+p \\ (E+p)v_2 \end{array}\right)=
	\left(\begin{array}{c} u_3 \\ \frac{u_2 u_3}{u_1} \\ \frac{u_3^2}{u_1} + p \\ \frac{u_3}{u_1} (u_4+p) \end{array}\right)
\end{align*}

Insgesamt ergeben sich die Matrizen $A=\frac{\partial f}{\partial u}$ und $B=\frac{\partial g}{\partial u}$

\begin{align*}
	A=
	\scalemath{0.8}{
		=\begin{pmatrix}
			0 & 1 & 0 & 0\\
			-(\frac{u_2}{u_1})^2- (\gamma -1) \left[\frac{u_2^2}{2 u_1^2}+\frac{u_3^2}{2 u_1^2}\right]    & \frac{2 u_2}{u_1} - (\gamma - 1)\frac{u_2}{u_1} &  (\gamma - 1) \frac{u_3}{u_1} & (\gamma - 1) \\
			-\frac{u_2 u_3}{u_1^2} & \frac{u_3}{u_1} & \frac{u_2}{u_1} & 0 \\
			-\frac{u_2 u_4}{u_1^2}-\frac{u_2}{u_1^2}(\gamma -1) \left(u_4+\frac{u_2^2+u_3^2}{2u_1}\right) - \frac{u_2}{u_1}(\gamma - 1) \frac{u_2^2+u_3^2}{2u_1^2} & 
			\frac{u_4}{u_1}+\frac{1}{u_1}(\gamma - 1) (u_4+ \frac{u_2^2+ u_3^2}{2 u_1}) + (\gamma - 1) (\frac{u_2}{u_1})^2
			& (\gamma - 1) \frac{u_2 u_3}{u_1^2} & \frac{u_2}{u_1} - (\gamma - 1) \frac{u_2}{u_1} 
	\end{pmatrix} }
\end{align*}

\begin{align*}
	B
	\scalemath{0.8}{
		=\begin{pmatrix}
			0 & 0 & 1 & 0\\
			-\frac{u_2 u_3}{u_1^2}  & \frac{u_3}{u_1} &   \frac{u_2}{u_1} & 0\\
			-(\frac{u_3}{u_1})^2- (\gamma -1) \left[\frac{u_2^2}{2 u_1^2}+\frac{u_3^2}{2 u_1^2}\right]  & (\gamma - 1 ) \frac{u_2}{u_1}& \frac{2 u_3}{u_1} + (\gamma - 1)\frac{u_3}{u_1} & ( \gamma - 1) \\
			-\frac{u_3  u_4}{u_1^2}-\frac{u_3}{u_1^2}(\gamma -1) \left(u_4+\frac{u_2^2+u_3^2}{2u_1}\right) - \frac{u_3}{u_1}(\gamma - 1) \frac{u_2^2+u_3^2}{2u_1^2} &  (\gamma - 1) \frac{u_2 u_3}{u_1^2}
			& \frac{u_4}{u_1}+\frac{1}{u_1}(\gamma - 1) (u_4+ \frac{u_2^2+ u_3^2}{2 u_1}) + (\gamma - 1) (\frac{u_3}{u_1})^2  & \frac{u_3}{u_1} - (\gamma - 1) \frac{u_3}{u_1} 
	\end{pmatrix} }
\end{align*}

Die Schallgeschwindigkeit ist gegeben durch $c=\sqrt{\frac{\gamma p}{\varrho}}$
Die Eigenwerte von A sind gegeben durch:
\begin{align*}
	\lambda_1=v_1 - c ,\lambda_2 = \lambda_3 = v_1 , \lambda_4 = v_1 + c
\end{align*}

Und die von B durch:
\begin{align*}
	\lambda_1=v_2- c ,\lambda_2 = \lambda_3 = v_2 , \lambda_4 = v_2 + c
\end{align*}

TODO: Noch einf�gen, warum max EV wichtig f�r Zeitint. Gute Begr�ndung aus alten Projekten nehmen!

\qed

\end{document}
























