



\documentclass[11pt]{scrartcl}

%-------------------------------------------------------------------------------------------
% Präambel
%-------------------------------------------------------------------------------------------

% Pakete zur Verwendung der europäischen Schriftarten sowie der deutschen Sprache
\usepackage[ngerman]{babel}
\usepackage[T1]{fontenc}
\usepackage[utf8]{inputenc}


% Pakete zur Vewendung mathematischer Umgebungen und Symbole
\usepackage{amsmath}
\usepackage{amssymb}
\usepackage{amsthm}

% Paket zum Einbinden von Graphiken und Bildern
\usepackage{graphicx}

% Paket zur Verwendung nummerierter Auflistungen
\usepackage{enumerate}


% Paket zum unmittelbaren Erzwingen von Abbildungspositionen
\usepackage{here}

% Paket zur Verwendung einer verbesserten Schriftart
\usepackage{lmodern}

% Paket zur besseren Implementierung von Code in den Fließtext
\usepackage{fancyvrb}

% Paket zur Verlinkung der Labels und des Inhaltsverzeichnisses
\usepackage[plainpages=false]{hyperref}


% Paket zum Verwenden gewisser mathematischer Symbole
\usepackage{mathtools}
\usepackage{stmaryrd} %ZB \rrbracket \llbracket

\usepackage{listings}
\usepackage{xcolor}

\usepackage{graphicx}

% Paket zur Verwendung nummerierter Auflistungen
\usepackage{enumerate}

% Paket zur Verwendung einer verbesserten Schriftart
\usepackage{lmodern}

% for matrices to fit the page
\newcommand\scalemath[2]{\scalebox{#1}{\mbox{\ensuremath{\displaystyle #2}}}}

\usepackage{siunitx}

% Umgebung fuer die Aufgabenbeschreibung
\newtheorem {Aufgabe}{Aufgabe}

% Formatierung der Seiten
\oddsidemargin=0.in
\topmargin=-1.5cm
\textheight=23cm
\textwidth=16cm


% Einrücken von neuen Zeilen ausschalten
\setlength{\parindent}{0cm}


\begin{document}

%-------------------------------------------------------------------------------------------
% Deckblatt
%-------------------------------------------------------------------------------------------

% Umnummerierung des Decksblatts zum fehlerfreien Gebrauch des hyperref- Pakets
\pagenumbering{Alph}

% Setzen von Titel, Autor und Datum auf dem Deckblatt
\title{{\Huge The Chipmunk Adventure} \\[18pt]
Wissenschaftliches Rechnen II \\[18pt]}
\author{ \textbf{Teamname} \\
Jan Breuer\\
Mats Lerho \\
Henrik Zunker}
\date{\today}

% Erzeugung des Deckblatt
\maketitle

% Weitere Angaben (Arbeitsgruppe, Institut, Universität, Betreuung usw.)
%\begin{center} 
%\begin{Large}
%Mathematisches Institut \\[3pt]
%Mathematisch-Naturwissenschaftliche Fakultät \\[3pt]
%Universität zu Köln \\[3cm]
%Betreuung: Prof. Dr.-Ing. Gregor Gassner
%\end{Large}
%\end{center}

% Unterdrücken der Seitennummerierung
\thispagestyle{empty}


%-------------------------------------------------------------------------------------------
% 
%-------------------------------------------------------------------------------------------

\newpage
\tableofcontents
\thispagestyle{empty}
\newpage

% Umstellung der Seitennummerierung auf arabische Ziffern
\pagenumbering{arabic}
\section{Meilenstein 1}
To have a good foundation for the subsequent milestones, we focus on the Euler equations which have the vector of conservative variables $u=(\rho, \rho v1, \rho v2, E)^T$. In particular, we concentrate on the implementation of the standard strongform DGSEM in $2$D for a Cartesian (not necessarily square shaped) block.
\subsection{Divergenzform}

1. Formulate the compressible Euler equations in divergence form 
\begin{align*}
	u_t+f(u)_x + g(u)_y =0 
\end{align*}
and explicitly state the fluxes f, g.
\\
\\
\textbf{Lösung:}

Die kompressiblen Euler Gleichungen sind gegeben durch:
\begin{align*}
	\varrho_t+ \nabla \  \circ \ (\varrho v)&=0\\
	(\varrho v)_t+ \nabla \  \circ \ (\varrho vv^T + pI)&=0\\
	E_t + \nabla \ \circ \ [(E+p)v]&=0
\end{align*}
Da $v=(v_1,v_2)^T$ gilt, sind die Gleichungen äquivalent zu
\begin{align*}
	\varrho_t+ (\varrho v_1)_x + (\varrho v_1)_y  &=0\\
	(\varrho v_1)_t+ (\varrho v_1^2 + p)_x + (\varrho v_1 v_2)_y &=0\\
	(\varrho v_2)_t+ (\varrho v_1 v_2 )_x + (\varrho v_2^2 + p)_y &=0\\
	E_t + [(E+p)v_1]_x + [(E+p)v_2]_y&=0
\end{align*}
Damit lässt sich die Divergenzform aufstellen:
\begin{align*}
	\left(\begin{array}{c} \varrho \\ \varrho v_1 \\ \varrho v_2 \\ E \end{array}\right)_t	+  \left(\begin{array}{c} \varrho v_1 \\ \varrho v_1^2 + p \\ \varrho v_1 v_2 \\ (E+p)v_1 \end{array}\right)_x +  \left(\begin{array}{c} \varrho v_2 \\ \varrho v_1 v_2 \\ \varrho v_2^2+p \\ (E+p)v_2 \end{array}\right)_y = 0
\end{align*}

\qed
\newline
\noindent 
\subsection{quasilineare Form}

2. Write the equations in their quasi-linear form
\begin{align*}
	u_t+Au_x + Bu_y =0 
\end{align*}
and find the eigenvalues of flux Jacobian matrices A, B. Explain why
the maximum eigenvalue is important in the explicit time integration.
\\
\\
\textbf{Lösung:}
Sei $u=(u_1,u_2,u_3,u_4)^T=(\varrho , \varrho v_1, \varrho v_2, E)^T$. \\
Dann gilt 
\begin{align*}
	p&=(\gamma -1)(E+\frac{\varrho}{2} ||v||^2)
	=(\gamma -1)(E+\frac{\varrho}{2} (v_1^2+v_2^2))
	=(\gamma -1)(u_4+ \frac{u_2^2}{2 u_1}+\frac{u_3^2}{2 u_1} )
\end{align*}
und 
\begin{align*}
	f=\left(\begin{array}{c} \varrho v_1 \\ \varrho v_1^2 + p \\ \varrho v_1 v_2 \\ (E+p)v_1 \end{array}\right)= \left(\begin{array}{c} u_2 \\ \frac{u_2^2}{u_1} + p \\ \frac{u_2 u_3}{u_1} \\ \frac{u_2}{u_1} (u_4+p) \end{array}\right)\\
	g=\left(\begin{array}{c} \varrho v_2 \\ \varrho v_1 v_2 \\ \varrho v_2^2+p \\ (E+p)v_2 \end{array}\right)=
	\left(\begin{array}{c} u_3 \\ \frac{u_2 u_3}{u_1} \\ \frac{u_3^2}{u_1} + p \\ \frac{u_3}{u_1} (u_4+p) \end{array}\right)
\end{align*}

Insgesamt ergeben sich die Matrizen $A=\frac{\partial f}{\partial u}$ und $B=\frac{\partial g}{\partial u}$

\begin{align*}
	A=
	\scalemath{0.8}{
		\begin{pmatrix}
			0 & 1 & 0 & 0\\
			-(\frac{u_2}{u_1})^2- (\gamma -1) \left[\frac{u_2^2}{2 u_1^2}+\frac{u_3^2}{2 u_1^2}\right]    & \frac{2 u_2}{u_1} - (\gamma - 1)\frac{u_2}{u_1} &  (\gamma - 1) \frac{u_3}{u_1} & (\gamma - 1) \\
			-\frac{u_2 u_3}{u_1^2} & \frac{u_3}{u_1} & \frac{u_2}{u_1} & 0 \\
			-\frac{u_2 u_4}{u_1^2}-\frac{u_2}{u_1^2}(\gamma -1) \left(u_4+\frac{u_2^2+u_3^2}{2u_1}\right) - \frac{u_2}{u_1}(\gamma - 1) \frac{u_2^2+u_3^2}{2u_1^2} & 
			\frac{u_4}{u_1}+\frac{1}{u_1}(\gamma - 1) (u_4+ \frac{u_2^2+ u_3^2}{2 u_1}) + (\gamma - 1) (\frac{u_2}{u_1})^2
			& (\gamma - 1) \frac{u_2 u_3}{u_1^2} & \frac{u_2}{u_1} - (\gamma - 1) \frac{u_2}{u_1} 
	\end{pmatrix} }
\end{align*}

\begin{align*}
	B =
	\scalemath{0.8}{
		\begin{pmatrix}
			0 & 0 & 1 & 0\\
			-\frac{u_2 u_3}{u_1^2}  & \frac{u_3}{u_1} &   \frac{u_2}{u_1} & 0\\
			-(\frac{u_3}{u_1})^2- (\gamma -1) \left[\frac{u_2^2}{2 u_1^2}+\frac{u_3^2}{2 u_1^2}\right]  & (\gamma - 1 ) \frac{u_2}{u_1}& \frac{2 u_3}{u_1} + (\gamma - 1)\frac{u_3}{u_1} & ( \gamma - 1) \\
			-\frac{u_3  u_4}{u_1^2}-\frac{u_3}{u_1^2}(\gamma -1) \left(u_4+\frac{u_2^2+u_3^2}{2u_1}\right) - \frac{u_3}{u_1}(\gamma - 1) \frac{u_2^2+u_3^2}{2u_1^2} &  (\gamma - 1) \frac{u_2 u_3}{u_1^2}
			& \frac{u_4}{u_1}+\frac{1}{u_1}(\gamma - 1) (u_4+ \frac{u_2^2+ u_3^2}{2 u_1}) + (\gamma - 1) (\frac{u_3}{u_1})^2  & \frac{u_3}{u_1} - (\gamma - 1) \frac{u_3}{u_1} 
	\end{pmatrix} }
\end{align*}

Die Schallgeschwindigkeit ist gegeben durch $c=\sqrt{\frac{\gamma p}{\varrho}}$
Die Eigenwerte von A sind gegeben durch:
\begin{align*}
	\lambda_1=v_1 - c ,\lambda_2 = \lambda_3 = v_1 , \lambda_4 = v_1 + c
\end{align*}

Und die von B durch:
\begin{align*}
	\lambda_1=v_2- c ,\lambda_2 = \lambda_3 = v_2 , \lambda_4 = v_2 + c
\end{align*}

\qed

TODO: Noch einfügen, warum max EV wichtig für Zeitint. Gute Begründung aus alten Projekten nehmen!

Um ein stabiles Verfahren zu erhalten, ist die Relation $\Delta t \sim \Delta x $ gemäß der CFL-Bedingung notwendig. Dabei wird die Systemgeschwindigkeit über den maximalen Eigenwert der System Matrizen, hier $A$ und $B$, abgeschätzt. 

\subsection{Standard DGSEM Verification}
3. Implement the standard strong form of the DGSEM for the compress-ible Euler equations on two-dimensional Cartesian meshes as you al-ready did in the previous course "Wissenschaftliches Rechnen I". Youcan also re-use the five stage, fourth order Runge-Kutta method aswell as the simple local Lax-Friedrichs Riemann solver. For now, the boundary conditions can be assumed to be periodic. \\
4. Verify the high-order convergence rate of the Standard DG version, as you did in the previous semester, for the two polynomial degrees N=3 and N=4. \\
\newline
\textbf{Lösung:} \\
Die Konvergenz des Verfahrens testen wir mit einer $2$-periodischen Sinusfunktion als Anfangsbedingung auf einem periodischen Gebiet $[0, 2]^2$ und von $t=0$ bis $t=2$.
\begin{align}
u_0(x_1, x_2, t) = 
\begin{pmatrix}
2 + 0.1 \cdot sin( \pi (x_1 + x_2 - t)) \\
2 + 0.1 \cdot sin( \pi (x_1 + x_2 - t)) \\
2 + 0.1 \cdot sin( \pi (x_1 + x_2 - t)) \\
2 + 0.1 \cdot sin( \pi (x_1 + x_2 - t))
\end{pmatrix}
\label{Anfangsbedingung Konvergenz}
\end{align}

Konvergenztabellen für das standard DGSEM.

\begin{table}[H]
\parbox{.45\linewidth}{
\centering
    \begin{tabular}{|r|r|r|}
    \hline\hline
    \textbf{Nq} & \textbf{Error} & \textbf{EOC} \\\hline
    2 & $\num{6.42854e-02}$ & $\num{}$ \\
    4 & $\num{1.32856e-02}$ & $\num{2.27}$ \\
    8 & $\num{8.78254e-04}$ & $\num{3.92}$ \\
    16 & $\num{4.49618e-05}$ & $\num{4.29}$ \\
    32 & $\num{2.92247e-06}$ & $\num{3.94}$ \\
    64 & $\num{1.91309e-07}$ & $\num{3.93}$ \\\hline\hline
  \end{tabular} 
  \caption{N = $3$}
  }
  \parbox{.45\linewidth}{
	\centering
    \begin{tabular}{|r|r|r|}
    \hline\hline
    \textbf{Nq} & \textbf{Error} & \textbf{EOC} \\\hline
    2 & $\num{2.38883e-02}$ & $\num{}$ \\
    4 & $\num{1.14360e-03}$ & $\num{4.38}$ \\
    8 & $\num{4.22035e-05}$ & $\num{4.76}$ \\
    16 & $\num{2.07943e-06}$ & $\num{4.34}$ \\
    32 & $\num{8.93401e-08}$ & $\num{4.54}$ \\
    64 & $\num{3.37353e-09}$ & $\num{4.73}$ \\\hline\hline
  \end{tabular}
   \caption{N = $4$}
  }
\end{table}

\newpage
\section{Meilenstein 2}
As we know, a scheme has to satisfy a certain amount of requirements toguarantee convergence to a physical correct and unique solution. In the lastsemester, we have constructed so-called entropy stable discretisations, whichmeans that the discrete mathematical entropy is a non-increasing quantity.These entropy stable discretisations satisfy the fundamental second law ofthermodynamics, which brings us closer to our goal to construct a physicalcorrect solution. As we have seen in the previous course, numerical experi-ments clearly demonstrate an improved robustness of the discretisation.For this portion of the project, we focus on the implementation of theDGSEM for the split form of the equations.
\subsection{Chandrashekar split form}
1. Replace the divergence form of the volume integral with the Chandrashekar split form (eq.  (3.20) in [1]) by exchanging the volumenumerical fluxes $F^\#$,$G^\#$. This way the Riemann solver also changes, though the data transfer does not. For the standard DGSEM the local Lax-Friedrichs method in the $x$-direction is 
\begin{align}
F^{\star} =\frac{1}{2}- \lambda_{max} \ 2 \ \llbracket \underline{u} \rrbracket
\end{align}
whereas for the split form approximations the Riemann solver is coupled to the split form
\begin{align}
 F^{\star} = F^\#_{\text{Chandrashekar}} - \lambda_{max} \ 2 \ \llbracket \underline{u} \rrbracket
\end{align}
2. Verify the high order convergence rate as well as the entropy-conservation and entropy-stabilisation properties of the split form scheme. \\
\newline
\textbf{Lösung:} \\
Die Konvergenz des Verfahrens testen wir erneut mit der $2$-periodischen Sinusfunktion aus Gleichung \ref{Anfangsbedingung Konvergenz} als Anfangsbedingung auf einem periodischen Gebiet $[0, 2]^2$ und von $t=0$ bis $t=2$.\\

Konvergenztabellen für das Split-Form DGSEM mit Chandrashekar Fluss.

\begin{table}[H]
\parbox{.45\linewidth}{
\centering
    \begin{tabular}{|r|r|r|}
    \hline\hline
    \textbf{Nq} & \textbf{Error} & \textbf{EOC} \\\hline
    2 & $\num{6.87889e-02}$ & $\num{}$ \\
    4 & $\num{1.42021e-02}$ & $\num{2.28}$ \\
    8 & $\num{8.05405e-04}$ & $\num{4.14}$ \\
    16 & $\num{3.93625e-05}$ & $\num{4.35}$ \\
    32 & $\num{2.49788e-06}$ & $\num{3.98}$ \\
    64 & $\num{1.91309e-07}$ & $\num{3.93}$ \\\hline\hline
  \end{tabular} 
  \caption{N = $3$}
  }
  \parbox{.45\linewidth}{
	\centering
    \begin{tabular}{|r|r|r|}
    \hline\hline
    \textbf{Nq} & \textbf{Error} & \textbf{EOC} \\\hline
    2 & $\num{2.29650e-02}$ & $\num{}$ \\
    4 & $\num{8.08062e-04}$ & $\num{4.83}$ \\
    8 & $\num{3.71735e-05}$ & $\num{4.44}$ \\
    16 & $\num{1.76278e-06}$ & $\num{4.40}$ \\
    32 & $\num{7.82467e-08}$ & $\num{4.49}$ \\
    64 & $\num{2.91605e-09}$ & $\num{4.75}$ \\\hline\hline
  \end{tabular}
   \caption{N = $4$}
  }
\end{table}
Zur Entropie Konservierung und Entropie Stabilität/Dissipativität: \\
Die Chandrashekar Split-Form DGSEM ist Entropie stabil/dissipativ und nicht Entropie erhaltend. 
Zum Test dieser Eigenschaft wurde die Anfangsbedingung einer schwachen Druckwelle gewählt und die Veränderung der Entropie von $t=0$ bis $t=2$ auf einem Gebiet mit periodischen Randbedingungen betrachtet.\\
Änderung der Entropie mit standard DGSEM und $N=3$: \\
totale Energie || Entropie $t=0$: $2.53146784e+00 \ || \ -5.92064714e-05$\\
totale Energie || Entropie $t=2$: $2.53146784e+00 \ || \ -3.64796886e-04$ \\
Änderung der Entropie mit Chendrashekar Split-Form DGSEM und $N=3$: \\
totale Energie || Entropie $t=0$: $2.53146784e+00 \ || \ -5.92064714e-05$\\
totale Energie || Entropie $t=2$: $2.53146784e+00 \ || \ -3.56391403e-04$ \\

\newpage
\section{Meilenstein 3}
We now want to approximate the solution of the Euler equations on curvi-linear geometries, which is especially important when our simulation domainhas curved boundaries. In the last semester, we learned how to constructcurvilinear mappings. For this part of the project, we will use curvilinearmappings to extend our standard and split-form DGSEM code to curvilinearstructured meshes. The first step is to make our code able to approximate the solution of the Eu-ler equations on a curvilinear block with periodic boundary conditions. Thecurvilinear block will be obtained from a Cartesian block, $ \tilde{x}, \tilde{y} \in [-1, 1]^2$, using the following mapping:


\subsection{Curved mesh, transfinite mapping and contravariant fluxes}
1. Generate the mesh: Implement routines to obtain the physical coordinates of the LGL nodes for all elements of the mesh.\\
2. Construct transfinite mappings to transform the coordinates of a reference element, $ \xi, \eta \in [-1, 1]^2$, to the physical coordinates of each element: $x(\xi,\eta)$. The input of the transfinite mapping routines shouldbe the physical coordinates of the LGL nodes of each element.\\
3. Rewrite the conservation law in terms of the contravariant fluxes,
\begin{align*}
  J u_t + \tilde{f}_\xi +\tilde{g}_\eta =0
\end{align*}

\textbf{Solution:}
The conservation law states 
\begin{align*}
 u_t + f_x +g_y &= 0 \\
 \Leftrightarrow  u_t + \frac{\partial f}{\partial\xi} \frac{\partial\xi}{\partial x}+\frac{\partial f}{\partial \eta} \frac{\partial \eta}{\partial y} + \frac{\partial g}{\partial\xi} \frac{\partial\xi}{\partial x}+\frac{\partial g}{\partial\eta} \frac{\partial\eta}{\partial y} &= 0 
\end{align*}
We use the derivations of the mapping to estimate the metric terms $\eta_x, \eta_y, \xi_x, \xi_y$:
\begin{align*}
\nabla x(\xi, \eta) &= 
\begin{pmatrix}
x_\xi & x_\eta\\
y_\xi & y_\eta
\end{pmatrix} \\ &=
\begin{pmatrix}
1 - 0.15 \frac{\pi}{2} sin(\frac{\pi}{2}\xi) cos(\frac{3\pi}{2} \eta) &
- 0.15 cos(\frac{\pi}{2}\xi) \frac{3\pi}{2} sin(\frac{3\pi}{2} \eta) \\
- 0.15 \cdot 2\pi sin(2\pi\xi) cos(\frac{\pi}{2} \eta) &
1 - 0.15 cos(2\pi \xi) \frac{\pi}{2}  sin(\frac{\pi}{2} \eta)
\end{pmatrix}
\end{align*}
With 

\end{document}



%  
%




















