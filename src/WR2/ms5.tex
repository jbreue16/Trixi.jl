



\documentclass[11pt]{scrartcl}

%-------------------------------------------------------------------------------------------
% Präambel
%-------------------------------------------------------------------------------------------

% Pakete zur Verwendung der europäischen Schriftarten sowie der deutschen Sprache
\usepackage[ngerman]{babel}
\usepackage[T1]{fontenc}
\usepackage[utf8]{inputenc}


% Pakete zur Vewendung mathematischer Umgebungen und Symbole
\usepackage{amsmath}
\usepackage{amssymb}
\usepackage{amsthm}

% Paket zum Einbinden von Graphiken und Bildern
\usepackage{graphicx}

% Paket zur Verwendung nummerierter Auflistungen
\usepackage{enumerate}


% Paket zum unmittelbaren Erzwingen von Abbildungspositionen
\usepackage{here}

% Paket zur Verwendung einer verbesserten Schriftart
\usepackage{lmodern}

% Paket zur besseren Implementierung von Code in den Fließtext
\usepackage{fancyvrb}

% Paket zur Verlinkung der Labels und des Inhaltsverzeichnisses
\usepackage[plainpages=false]{hyperref}


% Paket zum Verwenden gewisser mathematischer Symbole
\usepackage{mathtools}
\usepackage{stmaryrd} %ZB \rrbracket \llbracket

\usepackage{listings}
\usepackage{xcolor}

\usepackage{graphicx}

% Paket zur Verwendung nummerierter Auflistungen
\usepackage{enumerate}

% Paket zur Verwendung einer verbesserten Schriftart
\usepackage{lmodern}

% for matrices to fit the page
\newcommand\scalemath[2]{\scalebox{#1}{\mbox{\ensuremath{\displaystyle #2}}}}

\usepackage{siunitx}

% Umgebung fuer die Aufgabenbeschreibung
\newtheorem {Aufgabe}{Aufgabe}

% Formatierung der Seiten
\oddsidemargin=0.in
\topmargin=-1.5cm
\textheight=23cm
\textwidth=16cm


% Einrücken von neuen Zeilen ausschalten
\setlength{\parindent}{0cm}


\begin{document}

%-------------------------------------------------------------------------------------------
% Deckblatt
%-------------------------------------------------------------------------------------------

% Umnummerierung des Decksblatts zum fehlerfreien Gebrauch des hyperref- Pakets
\pagenumbering{Alph}

% Setzen von Titel, Autor und Datum auf dem Deckblatt
\title{{\Huge The Chipmunk Adventure} \\[18pt]
Wissenschaftliches Rechnen II \\[18pt]}
\author{ \textbf{Teamname} \\
Jan Breuer\\
Mats Lerho \\
Henrik Zunker}
\date{\today}

% Erzeugung des Deckblattes
\maketitle

% Weitere Angaben (Arbeitsgruppe, Institut, Universität, Betreuung usw.)
%\begin{center} 
%\begin{Large}
%Mathematisches Institut \\[3pt]
%Mathematisch-Naturwissenschaftliche Fakultät \\[3pt]
%Universität zu Köln \\[3cm]
%Betreuung: Prof. Dr.-Ing. Gregor Gassner
%\end{Large}
%\end{center}


\thispagestyle{empty} % Unterdrücken der Seitennummerierung


%-------------------------------------------------------------------------------------------
% 
%-------------------------------------------------------------------------------------------

\newpage
\tableofcontents
\thispagestyle{empty}
\newpage

% Umstellung der Seitennummerierung auf arabische Ziffern
\pagenumbering{arabic}

\section{Meilenstein 5}

\textbf{2.} Use an analytical function for u and corroborate that the DGSEM
discretization of q converges to the exact gradient when the mesh is
refined and the polynomial degree is increased.\\
\\
To show convergence for the DGSEM of the discretization of q, we use the function $u(x)=sin(\pi x)$ with exact solution $sol(x) = \pi cos(\pi x)$.

\begin{table}[H]
	\parbox{.45\linewidth}{
		\centering
		\begin{tabular}{|r|r|}
			\hline\hline
			\textbf{Nq} & \textbf{Error} \\\hline
			2 & $\num{3.6750704656983517}$   \\
			4 & $\num{0.8133516661362132}$  \\
			8 & $\num{0.16662357988381782}$ \\
			16 & $\num{0.025738783624845052}$ \\
			32 & $\num{0.0035138009422910343}$ \\
			64 & $\num{0.00044150523778929696}$ \\\hline
		\end{tabular}
		\caption{N = $3$}
	}
	\hspace{0.5cm}
	\parbox{.45\linewidth}{
		\centering
		\begin{tabular}{|r|r|}
			\hline\hline
			\textbf{Nq} & \textbf{Error} \\\hline
			2 & $\num{1.1364006136647853}$   \\
			4 & $\num{0.08089555578182672}$  \\
			8 & $\num{0.00418486214856717}$ \\
			16 & $\num{0.00013576365082590414}$ \\
			32 & $\num{5.283708624403971e-6}$ \\
			64 & $\num{1.6605103070332916e-7}$ \\\hline
		\end{tabular}
		\caption{N = $4$}
	}
\end{table}


\textbf{4.} Test free-stream preservation and EOCs using a fully periodic mesh.\\
\\
We test the free-stream preservation for the Navier-Stokes-Equation on on a cartesian grid with periodic BC.
We calculate the solution from $t=0$ to $t=1$ with $\mu = 0.001$..\\
Initial condition, conservative variables:
\begin{align*}
	\rho = 1.0 \\
  	\rho \ v_1 = 0.1 \\
  	\rho \ v_2 = -0.2 \\
  	rho \ e = 10.0
\end{align*}
The estimated $L_2$ error in each dimension \\
$ 7.14432553e-17 \\ 5.78703601e-16\\ 3.87269218e-16  \\ 1.28702835e-15$ \\

lies within the mechanical tolerance.\\

Convergency tables for our DGSEM are calculated with mu=0.0001.

\begin{table}[H]
\parbox{.45\linewidth}{
\centering
    \begin{tabular}{|r|r|r|}
    \hline\hline
    \textbf{Nq} & \textbf{Error} & \textbf{EOC} \\\hline
    2 & $\num{6.85535e-02}$ &  \\
    4 & $\num{1.42105e-02}$ & $\num{2.27}$ \\
    8 & $\num{8.12136e-04}$ & $\num{4.13}$ \\
    16 & $\num{4.90295e-05}$ & $\num{4.05}$ \\\hline\hline
  \end{tabular} 
  \caption{N = $3$}
  }
  \hspace{0.5cm}
  \parbox{.45\linewidth}{
	\centering
    \begin{tabular}{|r|r|r|}
    \hline\hline
    \textbf{Nq} & \textbf{Error} & \textbf{EOC} \\\hline
    2 & $\num{2.29716e-02}$ &  \\
    4 & $\num{8.07948e-04}$ & $\num{4.83}$ \\
    8 & $\num{3.71614e-05}$ & $\num{4.44}$ \\
    16 & $\num{1.76198e-06}$ & $\num{4.40}$ \\\hline\hline
  \end{tabular}
   \caption{N = $4$}
  }
\end{table}

Furthermore, we test our method for a Gaussian impuls in the densitiy with constant pressure as initial condition.
The initial condition for $p=1$ is given by:
\begin{align*}
	\rho = 1 + exp(-(x^2 + y^2)) / 2 \\
	\rho \ v_1 = 1 + exp(-(x^2 + y^2)) / 2 \\
	\rho \ v_2 = 1 + exp(-(x^2 + y^2)) / 2 \\
	rho \ e = p / (\gamma - 1) + 1 / 2 * \rho * (v_1^2 + v_2^2)
\end{align*}

A visualization of the runs up to t = 5 for the Compressible Euler and the Navier-Stokes equations are additionally included as Gif files in the appendix. 
In Figure \ref{gaust0} we see the condition at time t=0.
\begin{figure}[H]
	\centering
	\includegraphics{medien/gaus_t0_rho.png}
	\caption{Gaussian curve for $t=0$}
	\label{gaust0}
\end{figure}

 Now we look at the figure at time $t = 5$ and see that the Euler equation \ref{gaus_euler_t5} preserves the impuls thanks to the periodic BC. With the Navier-Stokes equation in figure \ref{gaus_nse_t5} we see well the influence of the parabolic terms. These soften the solution.
 
 \begin{figure}[H]
 	\centering
 	\includegraphics{medien/eulert5_rho.png}
 	\caption{Gaussian impuls in density for $t=5$ solved with compressible Euler Equation}
 	\label{gaus_euler_t5}
 \end{figure}

 \begin{figure}[H]
	\centering
	\includegraphics{medien/nset5_rho.png}
	\caption{Gaussian impuls in density for $t=5$ solved with Navier-Stokes Equation}
	\label{gaus_nse_t5}
\end{figure}




\newpage
\addcontentsline{toc}{section}{Literaturverzeichnis}
\bibliography{Literaturverzeichnis}
\bibliographystyle{abbrv}



\end{document}

